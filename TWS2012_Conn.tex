\documentclass[serif,mathserif]{beamer}
\usepackage{amsmath, amsfonts, epsfig, xspace, color, colortbl,natbib}
\usepackage{algorithm,algorithmic}
\usepackage{animate}
%\usepackage{pstricks,pst-node}
%\usepackage{multimedia}
\usepackage{movie15}
\usepackage[normal,tight,center]{subfigure}
\setlength{\subfigcapskip}{-.5em}
%\usepackage{beamerthemesplit}
%\usetheme{lankton-keynote}

%%%
% PRELIMINARY FORMATTING
%%%
\renewcommand\sfdefault{phv}
\renewcommand\familydefault{\sfdefault}
\usetheme{default}
\usepackage{color}
\useoutertheme{default}
\usepackage{texnansi}
\usepackage{marvosym}
\definecolor{bottomcolour}{rgb}{0.32,0.3,0.38}
\definecolor{middlecolour}{rgb}{0.08,0.08,0.16}
\definecolor{noaaturq}{RGB}{2,171,216}
\definecolor{noaablue}{RGB}{20,21,127}
\setbeamerfont{title}{size=\huge}
\setbeamercolor{structure}{fg=gray}
\setbeamertemplate{frametitle}[default]%[center]
\setbeamercolor{normal text}{bg=black, fg=white}
\setbeamercolor{block title}{parent=normal text, bg=noaaturq}%noaablue!60!white}
\setbeamercolor{block body}{parent=normal text, use=block title,bg=noaablue}%block title.bg!50!black}
\setbeamertemplate{background canvas}[vertical shading]
[bottom=bottomcolour, middle=middlecolour, top=black]
\setbeamertemplate{items}[circle]
\setbeamerfont{frametitle}{size=\huge}
\setbeamertemplate{navigation symbols}{} %no nav symbols
\setbeamertemplate{blocks}[rounded]
\graphicspath{{c:/users/paul.conn/git/hierarchicalDS/presentations/images/}}

%\usecolortheme{seahorse}


\begin{document}

\frame{
\centering
\renewcommand{\baselinestretch}{1.8}\normalsize
{\huge \textcolor{noaaturq}{Hierarchical models for wildlife transect surveys}}\\
\bigskip\bigskip
\renewcommand{\baselinestretch}{1.25}\normalsize
Paul B. Conn\\
\footnotesize \textcolor{lightgray}{{\em NOAA National Marine Mammal Laboratory\\
Seattle, Washington}}\\
\textcolor{lightgray}{Email: paul.conn@noaa.gov}\\
\bigskip\bigskip
\textcolor{lightgray}{TWS 2012\\
Portland, OR\\
October 17, 2012}\\	
\vspace*{\fill}
\begin{figure}
%	%\subfigure{\includegraphics[height=2cm]{UAFlogo.png}}
%	\hspace*{\fill}
%	\subfigure{\includegraphics[height=2cm]{noaa_logo.png}}
\hspace{\fill}
\includegraphics[height=2cm]{NOAA-logo.pdf}
\end{figure}
}

 \section{Outline}  % add these to see outline in slides

\begin{frame}
  \frametitle{Outline}
  \begin{itemize}
  \item Design-based inference \pause
  \item Model-based inference \pause
  \item Hierarchical modeling roadmap \pause
  \item Simulations \pause
  \item Seal example \pause %leave out the \pause on the final item
  \item Extensions and future work
  \end{itemize}
\end{frame}

\section{Introduction} % add these to see outline in slides

\begin{frame}
\frametitle{Design-based inference}
\begin{columns}[c]
\column{1.5in}
Transect surveys are often used to enumerate wildlife populations
\column{2in}
\framebox{\includegraphics[width=6cm]{basic_pop.png}}
\end{columns}
\end{frame}

\begin{frame}
\frametitle{Design-based inference}
\begin{columns}[c]
\column{1.5in}
Transect surveys are often used to enumerate wildlife populations
\column{2in}
\framebox{\includegraphics[width=6cm]{basic_pop_transects.png}}
\end{columns}
\end{frame}

\begin{frame}
  \frametitle{Design-based inference}
\begin{columns}[c]
\column{1.5in}
  \begin{block}{Design-based estimator}
  \huge $\hat{N}=\frac{n}{P_a}$ \large \\
  \vspace{1cm}
  $N=$ Abundance \\
  $n=$ \# of animals counted \\
  $P_a=$ proportion of area sampled
  \end{block}
\column{2in}
\framebox{\includegraphics[width=6cm]{basic_pop_transects.png}}
\end{columns}
\end{frame}

\begin{frame}
  \frametitle{Design-based inference}
\begin{columns}[c]
\column{1.5in}
If all animals are not detectable in our strip, we need to estimate detection probability, $P_d$. \vspace{.5cm}

{\color{noaaturq}$\Rightarrow$ One approach: relate detectability to distance from transect line}
\column{2in}
\framebox{\includegraphics[width=6cm]{basic_pop_distances.png}}
\end{columns}
\end{frame}

\begin{frame}
  \frametitle{Design-based inference}
\begin{columns}[c]
\column{1.5in}
 \begin{block}{Distance estimator (design-based)}
  \huge $\hat{N}=\frac{n}{P_a \hat{P}_d}$ \large \\
  \end{block}
  \vspace{.5 cm}
  \begin{center}
  \includegraphics[width=2cm]{distance_book.jpg}
  \end{center}
\column{2in}
\framebox{\includegraphics[width=6cm]{basic_distance_histogram.png}}
\end{columns}
\end{frame}

\begin{frame}
  \frametitle{Design-based inference}
  Large improvements to design-based abundance estimators over the last few decades
   \begin{itemize}
  \item Account for multiple covariates influencing detection probability \pause
  \item Model data from multiple observers to minimize assumption violations \pause
  \item \color{noaaturq} Difficult to make inferences about the effects of habitat covariates
  on underlying density (only possible with two step approach) \pause
  \item Requires preset sampling design \pause
  \item Requires static population \pause
  \item Difficult to incorporate spatial or temporal autocorrelation
  \end{itemize}
\end{frame}

\begin{frame}
  \frametitle{Model-based inference}
  \begin{columns}[c]
  \column{1.5in}
  \begin{block}{Challenges with wildlife surveys}
    \begin{itemize}
    \item Weather and logistics often influence
    where sampling occurs
    \item Abundance isn't static in time or space
    \end{itemize}
  \end{block}
  \column{2in}
  \framebox{\includegraphics[width=6cm]{heli_tracks.png}}
  \end{columns}
\end{frame}

\begin{frame}
  \frametitle{Model-based inference}
    Conceptualize data collected as having arisen from two conceptually
     distinct processes, providing a model for each:
    \begin{itemize}
    \item {\color{noaaturq} Process model} Describes how abundance or a surrogate
       (density, abundance intensity) varies over time or space \pause
    \item {\color{noaaturq} Observation model} Describes the data collection process - how underlying abundance is translated into data (realization of stochastic sampling process) \pause
    \end{itemize}
    {\color{noaaturq}Need to estimate the parameters of each model and predict abundance over a surface!}
\end{frame}

\begin{frame}
  \frametitle{Model-based inference}
  Types of model-based analyses for animal transect data
  \begin{columns}[t]
  \column{1.33in}
  \begin{block}{Thinned point process}
    \begin{itemize}
    \item Hedley and Buckland (\citeyear{HedleyBuckland2004})
    \item Johnson et al. (\citeyear{JohnsonEtAl2010}), R package {\tt dspat}
    \end{itemize}
  \end{block}
  \column{1.33in}
  \begin{block}{Highfalutin Poisson GLMM}
    \begin{itemize}
      \item Moore and Barlow (\citeyear{MooreBarlow2011})
      \item Chelgren et al. (\citeyear{ChelgrenEtAl2011})
    \end{itemize}
  \end{block}
  \column{1.33in}
  \begin{block}{Data augmentation}
    \begin{itemize}
      \item Royle and Dorazio (\citeyear{RoyleDorazio2008})
      \item Schmidt et al. (\citeyear{SchmidtEtAl2011})
      \item $^\dag$Conn et al. (\citeyear{ConnEtAl2012}), R package {\tt HierarchicalDS}
    \end{itemize}
  \end{block}
  \end{columns}
  $^\dag$Accounts for double observers
\end{frame}



\begin{frame}
  \frametitle{Introduction}
  \begin{center}
   \framebox{\includegraphics[width=10.5cm]{Bering_ice.png}}
   \end{center}
\end{frame}

\begin{frame}
  \frametitle{Introduction}
  {\color{noaaturq}
  An additional challenge: Partial observability and errors in species assignment
  }

  \vspace{.5cm}

  \begin{columns}[c]
  \column{3in}
  \begin{tabular}{cccc}
  \hline \hline
 Obs1 & Obs2 & Distance & Group size\\
  \hline
 Bearded & Bearded & 1 & 2\\
 Bearded & 0 & 5 & 1 \\
 Bearded & Unknown & 1 & 2 \\
 Unknown & Unknown & 2 & 3 \\
 Unknown & Unknown & 4 & 8 \\
 0 & Spotted & 4 & 1 \\
 Spotted & Bearded & 2 & 2 \\
  \hline
  \end{tabular}
  \column{1in}
  \includegraphics[width=1in]{Spotted.png}

  \includegraphics[width=1in]{Bearded.png}
  \end{columns}
\end{frame}


\begin{frame}
  \frametitle{Introduction}
 \begin{block}{Model-based approaches to estimation}
  \begin{itemize}
    \item Inference based on {\it prediction} (e.g. on spatial surface)
    \item Sampling design can focus on minimizing prediction error (regular placement of transects)
        but not strictly necessary
    \item Hierarchical extensions possible (spatial, temporal variation in abundance)
    \item Separation of process and observation models lends itself to modular approach to model building
  \end{itemize}
  \end{block}
\end{frame}

\begin{frame}
  \frametitle{Introduction}
 \begin{block}{Goals for modeling}
  Develop hierarchical framework for line transect sampling that permits
  \begin{itemize}
    \item Straightforward inclusion of temporal/spatial effects on abundance
    \item Multiple observers with possible observer dependence
    \item Estimation of individual covariate distributions (e.g. group size)
    \item Possible extensions for species misidentification/partial observation
  \end{itemize}
  \end{block}
\end{frame}

\section{Math}
\begin{frame}
  \frametitle{Math}
 \begin{columns}[c]
  \column{1.5in}
  \begin{block}{Start by discretizing time \& space}
  Let $D_1, D_2, \hdots, D_S$ form a partition of some area of interest, $D$.
  \end{block}
  \column{2in}
   \framebox{\includegraphics[width=6cm]{partition.jpg}}
   \end{columns}
\end{frame}

\begin{frame}
  \frametitle{Math}
  Consider separate processes for:
  \begin{enumerate}
  \item Spatial and temporal dynamics of abundance at cell level {\color{noaaturq} (Process model)} \pause
  \item Abundance in a specific transect/grid cell combination {\color{noaaturq} (Local abundance model)} \pause
  \item Observations in a particular cell at a particular time {\color{noaaturq} (Observation model)}
  \end{enumerate}
\end{frame}

\begin{frame}
  \frametitle{Math: Process model}
  Group abundance ($G_s$) is difficult to work with
  \begin{itemize}
    \item integer valued \pause
    \item dependent upon spatial support \pause
    \item possible disconnect with observation model (later) \pause
  \end{itemize}
 {\color{noaaturq} $\Rightarrow$ Assume that abundance is Poisson distributed, and work with the log of abundance intensity, $\nu_{st}$:}
 $$ G_{s} \sim {\rm Poisson}(\lambda_{s}),$$
 $$   \lambda_{s}=A_s \exp(\nu_{s}) $$
 where $\lambda_{s}$ gives abundance intensity in cell $s$, and $A_s$ gives area of cell $s$ proportional to mean cell area  \pause \\
 {\color{noaaturq} $\Rightarrow$ How does $\boldsymbol{\nu}$ change over time and space?}
 \end{frame}

\begin{frame}
  \frametitle{Math: Process model}
    {\color{noaaturq} One possibility (implemented in R package {\it hierarchicalDS}:)}
    $$
    \boldsymbol{\nu} \sim {\rm Normal}({\bf X}\boldsymbol{\beta}+\boldsymbol{\eta},\tau_\nu^{-1})
    , \hspace{2mm} {\rm where}
    $$
    ${\bf X}$ gives a design matrix, \\ \vspace{2mm}
    $\boldsymbol{\beta}$ is a vector of regression coefficients, \\ \vspace{2mm}
    $\boldsymbol{\eta}$ is a vector of spatially structured random effects (stay tuned), and \\ \vspace{2mm}
    $\tau_\nu$ is the precision associated with overdispersion relative to the Poisson distribution.\\
    \vspace{4mm}
    {\color{noaaturq} Other possibilities: cellular automata, resource selection models}
\end{frame}


\begin{frame}
  \frametitle{Math: Local abundance model}
    \begin{columns}[c]
    \column{2.5in}
      \framebox{\includegraphics[width=2.5in]{local_abundance.png}}
    \column{2.5in}
      \begin{itemize}
        \item Data are partitioned into grid cell/transect combinations \pause
        \item Area surveyed in each transect used to scale $\lambda_{s}$ to the actual area surveyed $\lambda_{st}=P_{st}\lambda_s$  \pause
        \item Posterior predictions can be generated for areas of cells not sampled: $G_{s,-t} \sim {\rm Poisson}\left( (1-\sum_t P_{st})\lambda_{st} \right)$ \pause
      \end{itemize}
   \end{columns}
   \vspace{.5in}
   {\color{noaaturq} (Note that $P_{st}$ could also be modified to incorporate availability!)}
\end{frame}

\begin{frame}
  \frametitle{Math: Local abundance model}
  For areas of cells that {\it are} sampled, update abundance based on a whole bunch of things (this is where the rubber meets the road):
      \begin{itemize}
        \item Predicted abundance, $P_{st}\lambda_{st}$ \pause
        \item Simultaneously estimated detection parameters (stay tuned) \pause
        \item Individual covariate values \& distributions (stay tuned) \pause
        \item Data augmentation with RJMCMC (e.g. Durban and Elston JABES 2005) \pause \\
        {\color{noaaturq} $\Rightarrow$ Model unobserved animals \& their covariates probabilistically}
      \end{itemize}
\end{frame}

\begin{frame}
  \frametitle{Math: Observation model (Data, finally!)}
        Link and Barker (2010) suggest the following observation model likelihood
        for distance data:
         $$
         [G_j^{\rm obs} | G_j, p_{ij}]=
        \left( \begin{array}{l}
        G_j \\
        G_j^{\rm obs}
        \end{array} \right)
        \prod_{i=1}^{G_j} p_{ij}^{Y_{ij}} (1-p_{ij})^{(1-Y_{ij})},
        $$
        where \\
        $p_{ij}$ is the probability of observing group $i$ while surveying transect $j$ \\
        $G_j$ is the number of animal groups in transect $j$ \\
        $G_j^{\rm obs}$ is the number of observed groups in transect $j$
\end{frame}

\begin{frame}
  \frametitle{Math: Observation model (Data, finally!)}
        For us, $p_{ij}$ is the probability that a group of animals is seen by {\it at least} one observer.  With two observers (and a probit link function):
         $$
        {\rm probit} \left(\begin{array}{c}
        				p_{ij1} \\
        				p_{ij2}
        		\end{array}\right) = \left[\begin{array}{c}
        				\tilde{Y}_{ij1} \\
        				\tilde{Y}_{ij2}
        		\end{array}\right] \sim {\rm MVN}
        \left( \left[ \begin{array}{c}
        				{\bf X}_{ij1}^{\rm det} \boldsymbol{\beta}^{\rm det}\\
        				{\bf X}_{ij2}^{\rm det} \boldsymbol{\beta}^{\rm det}
        		\end{array}\right],
                \left[ \begin{array}{cc}
                    1 & \rho_{ij} \\
                    \rho_{ij} & 1
                \end{array} \right]
        \right),
        $$
        $Y_{ijk}=1$ iff $\tilde{Y}_{ijk}>0$,
       $$
          p_{ij} = \int_0^\infty \int_0^\infty {\rm MVN}
            \left( \left[ \begin{array}{c}
    				x\\
    				y
    		\end{array} \right]; \left[
            \begin{array}{c}
    				{\bf X}_{ij1}^{\rm det} \boldsymbol{\beta}^{\rm
                    det}\\
    				{\bf X}_{ij2}^{\rm det} \boldsymbol{\beta}^{\rm
                    det}
    		\end{array}\right],
            \left[ \begin{array}{cc}
                1 & \rho_{ij} \\
                \rho_{ij} & 1
            \end{array} \right]
            \right) dx dy
        $$
\end{frame}

\begin{frame}
  \frametitle{Math: Observation model (Data, finally!)}
  Observer dependence: $\rho_{ij} =$ f(distance), with
  y-intercept constrained to be 0 [the point independence assumption], and maximum $\le 1$.\pause

  \begin{columns}[c]
  \column{1.8in}
  \vspace{.2cm}

   In {\tt HierarchicalDS},\\

   \vspace{.2cm}
  discrete distance:
    $$
    \rho_{ij}=\frac{(d_{ij}-1)}{\max(d_{ij}-1)},
    $$

    \vspace{.5cm}
  continuous distance:
    $$
    \rho_{ij}=\frac{d_{ij}}{\max(d_{ij})}.
    $$
  \column{2in}
     \framebox{\includegraphics[width=2in]{point_indep.png}}
  \end{columns}
\end{frame}

\begin{frame}
  \frametitle{Math: Observation model (Data, finally!)}
  Why choose a probit model?
  \begin{itemize}
    \item Constrains detection probability to (0,1) \pause
    \item Efficient Gibbs sampling (Albert \& Chib 1993 JASA). With
        one observer,
        $$
        \tilde{Y}_{ijk} \sim \textrm{Normal} \left( [{\bf X}^{\rm det} \boldsymbol{\beta}^{\rm det}]_{ijk},1  \right)
        $$
        where $\tilde{Y}_{ijk}<0$, while $\tilde{Y}_{ijk}>0$ if $Y_{ijk}=1$.
        {\color{noaaturq} $\Rightarrow$ Simulate $\tilde{\bf Y}$ directly using
        truncated normal pdfs.} \pause \\

        \hspace{.2cm}

        $$
        [\boldsymbol{\beta}^{\rm det}|\hdots]=\textrm{Normal}\left( ({\bf X}^\prime {\bf X})^{-1}{\bf X}^\prime \tilde{{\bf Y}},
        ({\bf X}^\prime {\bf X})^{-1} \right).
        $$

  \end{itemize}
\end{frame}

\begin{frame}
  \frametitle{Math: Covariate models}
  \begin{block}{Individual covariate pdfs, $g(\theta)$, in {\tt HierarchicalDS}}
  \begin{itemize}
    \item Poisson
    \item Poisson-lognormal mixture
    \item Zero-truncated Poisson
    \item Zero-truncated Poisson-lognormal mixture
    \item Categorical
    \item Normal
  \end{itemize}
  \end{block}
\end{frame}


\begin{frame}
    \frametitle{Math: Estimation}
    RJMCMC algorithm:
    \begin{enumerate}
        \item Update (add or subtract)  latent animals \pause
            \begin{itemize}
              \item Propose increase or decrease with probability 0.5 \pause
              \item Propose Num to increase/decrease from U(0,a) distribution \pause
              \item Propose individual covariate values from $g(\theta)$ \pause
              \item Accept joint proposal according to MH ratio (posterior proportional to a product of local abundance, observation, covariate models and
                  prior distributions) \pause
            \end{itemize}
        \item Update covariates for unobserved animals according to $[g(\theta)|Y=0]$ \pause
        \item Simulate $\tilde{Y}$ values for new additions \pause
    \end{enumerate}
    Other parameters updated via Gibbs or Metropolis-Hastings steps
\end{frame}

\begin{frame}
  \frametitle{RJMCMC Algorithm}
         \begin{tabular}{ccccccc}
        \hline \hline
        $Y_1$ & $Y_2$ & $\tilde{Y}_1$ & $\tilde{Y}_2$ & Distance & Species & Gr size\\
        \hline
        1 & 1 & 2.3 & 2.0 & 1 & A & 2\\
        1 & 0 & 1.0 & -0.4 & 3 & A & 1 \\
        1 & 1 & 0.6 & 0.3 & 4 & A & 1 \\
        0 & 0 & -0.5 & -0.8 & 3 & A & 2 \\
        0 & 0 & -1.2 & -1.6 & 5 & A & 1 \\
         \hline
        \end{tabular}
\end{frame}

\begin{frame}
  \frametitle{RJMCMC Algorithm}
         \begin{tabular}{ccccccc}
        \hline \hline
        $Y_1$ & $Y_2$ & $\tilde{Y}_1$ & $\tilde{Y}_2$ & Distance & Species & Gr size\\
        \hline
        1 & 1 & 2.3 & 2.0 & 1 & A & 2\\
        1 & 0 & 1.0 & -0.4 & 3 & A & 1 \\
        1 & 1 & 0.6 & 0.3 & 4 & A & 1 \\
        0 & 0 & -0.5 & -0.8 & 3 & A & 2 \\
        0 & 0 & -1.2 & -1.6 & 5 & A & 1 \\
        \rowcolor{noaaturq}
        0 & 0 & NA & NA & 3 & A & 2 \\
        \rowcolor{noaaturq}
        0 & 0 & NA & NA & 1 & A & 1 \\
        \rowcolor{noaaturq}
        0 & 0 & NA & NA & 5 & A & 1 \\
         \hline
        \end{tabular}
\end{frame}

\begin{frame}
  \frametitle{RJMCMC Algorithm}
         \begin{tabular}{ccccccc}
        \hline \hline
        $Y_1$ & $Y_2$ & $\tilde{Y}_1$ & $\tilde{Y}_2$ & Distance & Species & Gr size\\
        \hline
        1 & 1 & 2.3 & 2.0 & 1 & A & 2\\
        1 & 0 & 1.0 & -0.4 & 3 & A & 1 \\
        1 & 1 & 0.6 & 0.3 & 4 & A & 1 \\
        0 & 0 & -0.5 & -0.8 & 3 & A & 2 \\
        0 & 0 & -1.2 & -1.6 & 5 & A & 1 \\
        0 & 0 & NA & NA & \cellcolor{noaaturq} 2 & A & \cellcolor{noaaturq} 1 \\
        0 & 0 & NA & NA & \cellcolor{noaaturq} 4 & A & \cellcolor{noaaturq} 1 \\
        0 & 0 & NA & NA & \cellcolor{noaaturq} 5 & A & \cellcolor{noaaturq} 1 \\
         \hline
        \end{tabular}
\end{frame}

\begin{frame}
  \frametitle{RJMCMC Algorithm}
         \begin{tabular}{ccccccc}
        \hline \hline
        $Y_1$ & $Y_2$ & $\tilde{Y}_1$ & $\tilde{Y}_2$ & Distance & Species & Gr size\\
        \hline
        1 & 1 & 2.3 & 2.0 & 1 & A & 2\\
        1 & 0 & 1.0 & -0.4 & 3 & A & 1 \\
        1 & 1 & 0.6 & 0.3 & 4 & A & 1 \\
        0 & 0 & -0.5 & -0.8 & 3 & A & 2 \\
        0 & 0 & -1.2 & -1.6 & 5 & A & 1 \\
        0 & 0 & \cellcolor{noaaturq} -0.6 & \cellcolor{noaaturq} -0.2 & 2 & A & 1 \\
        0 & 0 & \cellcolor{noaaturq} -1.0 & \cellcolor{noaaturq} -1.8 & 4 & A & 1 \\
        0 & 0 & \cellcolor{noaaturq} -3.7 & \cellcolor{noaaturq} -2.8 & 5 & A & 1 \\
         \hline
        \end{tabular}
\end{frame}

\begin{frame}
  Well great, but how does this work in practice?
  \begin{enumerate}
    \item Simulated data, no spatial structure
    \item Known population of golf tees (no spatial structure)
    \item Simulated data (Matern process), spatial structure
  \end{enumerate}
\end{frame}

\section{Simulations I}
\begin{frame}
  \frametitle{Simulated data: no spatial structure}
  \begin{block}{Simulation study}
  \begin{itemize}
  \item Simulated data for two species: (a) linearly increasing trend in abundance as a function of a covariate, and (b) abundance as a quadratic function of a covariate.
  \item Group size simulated from a zero-truncated Poisson distribution
  \item Detection probability a function of observer (categorical), distance (categorical; 5 bins), group size (continuous), and species (categorical).
  \item Correlation in probit-scale responses of each observer linearly increasing from 0 to a maximum of 0.5 as a function of distance
  \item Two MCMC chains of length 270,000 with first 20,000 of each discarded as burnin
  \end{itemize}
  \end{block}
\end{frame}

\begin{frame}
  \frametitle{Simulated data: no spatial structure}
  Directed, acyclic graph
  \framebox{\includegraphics[width=3.5in]{Dag_simple}}
\end{frame}

\begin{frame}
  \frametitle{Simulated data: no spatial structure}
  Results: Posteriors (truth in red)
  \framebox{\includegraphics[width=2.7in]{sim_2pop_Ests}}
\end{frame}

\begin{frame}
  \frametitle{Simulated data: no spatial structure}
  Results: Abundance by transect
  \framebox{\includegraphics[width=2.7in]{sim_2pop_N_byTransect.png}}
\end{frame}

\section{Golf tees}
\begin{frame}
  \frametitle{Golf tee example}
  Example II: A `real' population
  \begin{itemize}
  \item Known population of golf tees at University of St. Andrews
  \item Random placement within two strata (density higher in Strata 2)
  \item Tee clusters differed by \# of tees (Poisson distributed), color, and exposure
  \item Surveyed by two independent groups of students using line transect design
  \item Analyzed extensively to demonstrate distance sampling methods in several text books (Borchers et al. 2002 Estimating Animal Abundance, Laake \& Borchers chapter in Advanced Distance Sampling)
  \item Detection probability modeled as a function of group size, tee color, exposure
  \end{itemize}
\end{frame}

\begin{frame}
  \frametitle{Golf tee example}
  \begin{columns}[c]
  \column{1.8in}
    \begin{itemize}
      \item Black - Yellow tees \\
            Gray - Green tees
      \item Square - `Exposed' \\
            Triangle - `Not exposed' \\
      \item Group size (1-8) - size of symbol \\
      \item Closed - Observed \\
      \item Open - Not observed\\
    \end{itemize}
  \column{2.5in}
      \framebox{\includegraphics[width=2.5in]{golf_tee}}
  \end{columns}
\end{frame}

\begin{frame}
  \frametitle{Golf tee example}
  Results: Abundance by color \& exposure (truth in red; Laake et al. estimates in blue)

  \framebox{\includegraphics[width=2.5in]{golf_tee_post_N}}
\end{frame}

\begin{frame}
  \frametitle{Golf tee example}
  Results: True and estimated group sizes
  \framebox{\includegraphics[width=2.5in]{golf_tee_grp_size.png}}
\end{frame}

\section{Simulations II}
\begin{frame}
  \frametitle{Simulated data with spatial structure}
  Example III: Patchy population
  \begin{itemize}
  \item Simulated abundance on a grid using a Matern process
  \item Transect placement determined according to a generalized random-tessellation stratified design (spatially balanced sampling)
  \item Detection a function of distance from centerline (categorical, 5 levels), observer ID (categorical, 3 levels), and group size (continuous)
  \item Intrinsic conditionally autoregressive (ICAR) model (Gaussian Markov Random Field)
      used to account for spatial dependence
  \end{itemize}
\end{frame}

\begin{frame}
  \frametitle{Simulated data with spatial structure}
  DAG for Example III: Patchy population
  \framebox{\includegraphics[width=3.5in]{DAG_spatial}}
\end{frame}

\begin{frame}
  \frametitle{Simulated data with spatial structure}
  \begin{columns}[c]
  \column{2in}
  Truth \vspace{.5cm}

  \framebox{\includegraphics[width=2in]{matern_truth.png}}
  True $N=148800$
  \column{2in}
  Mean posterior predictions \vspace{.5cm}

  \framebox{\includegraphics[width=2in]{matern_nosrr_mean.png}}
  $\hat{N}_1=304000$,$\hat{N}_2=2418000$
  \end{columns}
\end{frame}

\begin{frame}
  \frametitle{Simulated data with spatial structure}
  Yikes!!! What happened?
      \begin{itemize}
        \item Model ``trying" to fit observed data
        \item Sharp breaks in abundance only reconcilable with high variance
            associated with spatial random effects
        \item Multiple spatial patterns may fit the data nicely
        \item Nothing really holding abundance back in unsampled cells
      \end{itemize}
      {\color{noaaturq}
        $\Rightarrow$ Predicted abundance over the landscape biased high \\
        $\Rightarrow$ Need to provide more structure on spatial random effects! }
\end{frame}

\begin{frame}
  \frametitle{Simulated data with spatial structure}
      One approach (Hughes \& Haran ArXiv 1101.6649v1 [stat.ME]):
      \begin{itemize}
  \item Calculate the residual projection matrix, ${\bf P}^\bot = {\bf I}-{\bf X}({\bf X}^\prime{\bf X})^{-1}{\bf X}^\prime$.
 \item Determine the Moran operator matrix, $\boldsymbol{\Omega}=S {\bf P}^\bot {\bf C}{\bf P}^\bot / {\rm sum}({\bf C})$.
 \item Determine the eigenvalues, $\boldsymbol{\lambda}$, and eigenvectors, ${\bf V}$, of $\boldsymbol{\Omega}$.
 \item Use a criterion on $\boldsymbol{\lambda}$ to limit the number of ``effective" spatial random effects.  For instance, limiting ${\bf V}$ to those for which accompanying eigenvalues are greater than $p$.
 \item Reassemble the selected eigenvectors from ${\bf V}$ into a new, reduced dimensional matrix ${\bf K}$.
 \item Calculate $\boldsymbol{\eta}={\bf K}\boldsymbol{\theta}$, where $[\boldsymbol{\theta}|\tau_{\eta}]= {\rm MVN}({\bf 0},{\bf K}^\prime {\bf Q} {\bf K})$.      \end{itemize}
 \end{frame}

\begin{frame}
  \frametitle{Simulated data with spatial structure}
  \begin{columns}
  \column{2in}
  Truth \vspace{.5cm}

  \framebox{\includegraphics[width=2in]{matern_truth.png}}
  True $N=148800$

  \column{2in}
  Mean posterior predictions \vspace{.5cm}

  \framebox{\includegraphics[width=2in]{matern_srr8_mean.png}}
  $\hat{N}_1=158000$,$\hat{N}_2=168000$
  \end{columns}
\end{frame}

\begin{frame}
    \frametitle{Spatial prediction}
    \begin{itemize}
      \item Collecting covariates thought to influence animal density important!!
      \item Example illustrates possible issues with ``estimating" abundance via posterior prediction when there are large levels of residual autocorrelation: positive bias
      \item Spatial smoothing may help in these cases
    \end{itemize}
\end{frame}

\section{Final bits}
\begin{frame}
    \frametitle{Future extensions and final thoughts}
    \begin{itemize}
      \item I hope I've convinced you of the utility of using hierarchical models for
      transect data (with multiple observers or otherwise)
      \item Lots of room left for future work!
      \item Species misidentification
    \end{itemize}
\end{frame}

\begin{frame}
    \frametitle{Future extensions and final thoughts}
    \framebox{\includegraphics[width=3in]{DAG_misID}}
\end{frame}

\begin{frame}
  \frametitle{Acknowledgments}
  \begin{block}{Collaborators}
    \begin{itemize}
        \item Method development: Devin Johnson, Jeff Laake
        \item Spatial methods: Devin Johnson, Mevin Hooten, Jay ver Hoef
        \item Seal data: Michael Cameron, Peter Boveng, PEP Program at NMML
        \item Species MisID: Brett McClintock
        \item Golf tee data: David Borchers
    \end{itemize}
  \end{block}
\end{frame}

\begin{frame}
    \framebox{\includegraphics[width=3in]{questions}}
\end{frame}

\begin{frame}
  \frametitle{References}
  \tiny
  \bibliographystyle{plainnat}
  \bibliography{../manuscript/master_bib}
\end{frame}

\end{document}
